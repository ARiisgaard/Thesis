\chapter{This is a Chapter}\label{Chapter Ref}

\section{This is a section}

\subsection{This is a subsection}

\subsection*{This is a numberless sebsection}%You can remove the numbers by adding a *


\subsubsection{This is a subsubsection}



\textbf{This is in bold (ctrl+b)}


\textit{This is italic (ctrl+i)} 

\begin{itemize}
    \item This 
    \item is
    \item a
    \item list
\end{itemize}


\section{References \& sources}
This is a reference to chapter \ref{Chapter Ref} \\
This is a reference to figure \ref{Bird} \\
This is a reference to tabular \ref{my-label}\\
This is a reference to math \ref{Line2} \\

If you click any of them in the pdf it jumps to the relevant page


\textbf{Sources}

This is one way to present a source \citet{isover} 

This is another way \citep{isover}

References can be made at https://truben.no/latex/bibtex/ 



\section{Math}

\subsection{Math in text}
For math in text do it like this: 

Bla bla CO$_2$ bla bla

$\left(x+a\right)^n=\sum_{k=0}^{n}{\binom{n}{k}x^ka^{n-k}}$

Since math is a mess in latex it is easier to write it in Word, change it to  linear and copy it over


\subsection{Math outside text}

Math outside text look like this:

\begin{align}
    x = \frac{-b\pm\sqrt{b^2-4ac}}{2a}  \label{Line1}  \\
    y = 2\cdot \pi \label{Line2}
\end{align}

You can use the labels to reference to individual lines. Like if you want point out that line \ref{Line1} has nothing to do with line \ref{Line2}
\begin{align*}
    x = \frac{-b\pm\sqrt{b^2-4ac}}{2a} \\
    y = 2\cdot \pi
\end{align*}

If you don't want to make references you can add a * to remove the numbers

\begin{align}
    x &= \frac{-b\pm\sqrt{b^2-4ac}}{2a} \\
    y &= 2\cdot \pi
\end{align}

You can add a \& in front of equation marks in order to align the lines 

\section{Comments}

This is one way to write comments. Latex is printing a list in the end with all of there notes
\fxnote{This is a comment}

This is another way

\section{Python}

\begin{lstlisting}[language=iPython, caption={My Caption}, label= Test2,escapechar=|]
import os |\label{sargtewthdsgh}|
filename = "c:/users/data.txt" |\label{DetteErEnTest}|
fileobj = file(filename, "r") #Reads the file 

#Defines counters for the number of outliers and nonoutliers
NumberO= 0
NumberP= 0

InputFile = os.path.basename(filename) #Finds the name of the selected file
InputFileMinusFiletype= InputFile[:-4] #Removes the filetype

f= open("C:\Users\Public\Python/"+"outlier_"+InputFileMinusFiletype+".txt","w+") #Create the file for storing the outliers - the w+ creates the file, if it doesn't exist

a= open("C:\Users\Public\Python/"+"pts_"+InputFileMinusFiletype+".txt","w+")


for line in fileobj.readlines():
    linevalues = line.split(" ")#Tells the program how it should determent the difference between the lines
    Zval = float(linevalues[2]) #Tells the program to look at the third value, which is the z values
    if Zval > 80.0 or Zval < 75.0: #Defines what an outlier is
        f.write(line) #all outlier are added to the outlier file
        NumberO += 1 #This is used to count the total number of outliers
    else:
        a.write(line) #all other points are added to the other file
        NumberP += 1
    #do something with line
fileobj.close()

print "Number of outliers: "+ str(NumberO)
print "Number of points: "+ str(NumberP)
\end{lstlisting}

How to reference to line \ref{DetteErEnTest} in listing \ref{Test2}.


$f\left(x\right)=a_0+\sum_{n=1}^{\infty}\left(a_n\cos{\frac{n\pi x}{L}}+b_n\sin{\frac{n\pi x}{L}}\right)$