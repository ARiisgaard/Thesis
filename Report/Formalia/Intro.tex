\chapter{Introduction}

In 2019 the population in the world reached 7.7 billion people, which is an increase of one billion over the past twelve years. According to The United Nations Department of Economic and Social Affairs’ (UN DESA) median scenario the growth is expected to continue reaching 9.7 billon in 2050.
https://population.un.org/wpp/Publications/Files/WPP2019\_Highlights.pdf

To be able to adapt infrastructures to this population growth it is necessary to predict where these people will settle. While UN DESA provides this information on a national level (https://population.un.org/wpp/Download/Standard/Population/ - name: total population), it is more ideal with a more nuanced picture, since most planning are based on local or regional scale spatial projections.
https://iopscience.iop.org/article/10.1088/1748-9326/11/8/084003/meta

Other researchers (SEDAC, CISC) have used simulations to distribute the population within each country as raster layers. However due to the high resolution and/or small scales, visually comparing these raster datasets is a time-consuming task. The purpose of this project is to create a tool allowing fast and easy comparison of such raster datasets, focusing on the use case of population projections.

%Evt noget om at det vil være ekstra interessant at have områderne med stor vækst som case - tilføj her, hvis der skal argumenteres for en case senere

\section{Problem statement}

To explore the possibilities for creating such a comparison tool the following research question have been defined:

\textit{How can population rasters be visualized and compared efficiently and effectively?}

This broad main question will be answered by answering the following three subquestions:

\textbf{Which conventions exist for visualization of population projections?}

\textbf{Which functionalities are relevant for comparing different rasters?}

\textbf{How can a responsive user experience be ensured, when loading and visualizing large raster dataset?}

\fxnote{Maybe write something about the limitation - eg. that user testing wasn't a possibility due to time}


\section{Limitations}

\section{Report structure}

\fxnote{ADD: quick overview of what the solution is going to be, what is population projections, SSP}
The report have been divided into three parts. The first part is the literature review, which will address the first two subquestions and also present the two projections visualised in this project. Chapter x explores which conventions there exist for population projections, while the relevant functionalities for raster comparison are detailed in chapter x. Lastly chapter x will give an overview of the population projection SEDAC and CICS, which will be used as case for comparison.

The second part is addressing the last subquestion. First there is a definition of how a "responsive user experience" has been defined. Then different methods of visualising raster datasets are being tested in chapter x. Based on these initial tests a method will chosen, which will be evaluated in the next section.

The last part starts with a discussion in chapter x of the results of the previous part. This is then followed by the last two chapters x and x, which are the conclusion and future work.  

%Part I: Litterature review
%- Conventions, what are relevant functionalities, Explaining the two case projections
%
%Part II: Choice of method
%- Test of different methods 
%
%Part III: Discussion 